\documentclass{article}
\usepackage{listings}
\usepackage{xcolor}
\usepackage{geometry}
\geometry{a4paper, margin=1in}

\title{Python Script for Generating Report Cards}
\author{}
\date{}

\definecolor{codegray}{rgb}{0.5,0.5,0.5}
\definecolor{codepurple}{rgb}{0.58,0,0.82}
\definecolor{backcolour}{rgb}{0.95,0.95,0.92}

\lstdefinestyle{mystyle}{
    backgroundcolor=\color{backcolour},
    commentstyle=\color{codegray},
    keywordstyle=\color{blue},
    numberstyle=\tiny\color{codegray},
    stringstyle=\color{codepurple},
    basicstyle=\ttfamily\footnotesize,
    breaklines=true,
    captionpos=b,
    keepspaces=true,
    numbers=left,
    numbersep=5pt,
    showspaces=false,
    showstringspaces=false,
    showtabs=false,
    tabsize=2
}

\lstset{style=mystyle}

\begin{document}

\maketitle

\section*{Python Script}

\begin{lstlisting}[language=Python, caption=Python Script for Generating Report Cards]
import pandas as pd
from reportlab.lib.pagesizes import A4
from reportlab.lib import colors
from reportlab.platypus import SimpleDocTemplate, Table, TableStyle, Paragraph, Spacer
from reportlab.lib.styles import getSampleStyleSheet
import os

def generate_report_cards(file_path):
    try:
        # Read the Excel file
        df = pd.read_excel(file_path)

        # Validate required columns
        required_columns = {'Student ID', 'Name', 'Subject', 'Score'}
        if not required_columns.issubset(df.columns):
            raise ValueError(f"Excel file must contain columns: {required_columns}")

        # Handle missing or invalid data
        df.dropna(subset=['Student ID', 'Name', 'Subject', 'Score'], inplace=True)
        df['Score'] = pd.to_numeric(df['Score'], errors='coerce')
        df.dropna(subset=['Score'], inplace=True)

        # Group by Student ID and Name
        grouped = df.groupby(['Student ID', 'Name'])

        for (student_id, name), group in grouped:
            total_score = group['Score'].sum()
            average_score = group['Score'].mean()
            subject_scores = group[['Subject', 'Score']].values.tolist()

            # Create PDF
            file_name = f"report_card_{student_id}.pdf"
            doc = SimpleDocTemplate(file_name, pagesize=A4)
            elements = []
            styles = getSampleStyleSheet()

            # Title and Summary
            elements.append(Paragraph(f"Report Card for {name}", styles['Title']))
            elements.append(Spacer(1, 12))
            elements.append(Paragraph(f"Total Score: {total_score}", styles['Normal']))
            elements.append(Paragraph(f"Average Score: {average_score:.2f}", styles['Normal']))
            elements.append(Spacer(1, 12))

            # Subject-wise Scores Table
            table_data = [['Subject', 'Score']] + subject_scores
            table = Table(table_data)
            table.setStyle(TableStyle([
                ('ALIGN', (0, 0), (-1, -1), 'CENTER'),
                ('GRID', (0, 0), (-1, -1), 1, colors.black)
            ]))
            
            elements.append(table)

            # Build PDF
            doc.build(elements)
            print(f"Generated report card for {name} (ID: {student_id})")
    except Exception as e:
        print(f"Error: {e}")

if __name__ == "__main__":
    generate_report_cards("student_scores.xlsx")
\end{lstlisting}

\section*{Approach Explanation}

\begin{itemize}
    \item \textbf{Data Loading:} Reads the Excel file containing student scores using \texttt{pandas}.
    \item \textbf{Data Validation:} Ensures required columns exist and handles missing or invalid data.
    \item \textbf{Grouping and Calculation:} Groups data by student and calculates total and average scores.
    \item \textbf{PDF Generation:} Uses \texttt{ReportLab} to create a PDF report for each student, including a table of subject-wise scores.
    \item \textbf{Error Handling:} Captures and prints errors during the process for debugging.
\end{itemize}

\end{document}
